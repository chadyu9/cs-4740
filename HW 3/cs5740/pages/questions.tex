\question{1}{The first model of homework 3 is a task-specific discriminative model. List two limitations of the task-specific discriminative model.}

\answer{}

%%%%%%%%%%%%%%%%%%%%%%%%%%%%%%%%%%%%%%%%%%%%%%

\question{2}{Below is the augmented natural language designed for NER task. Is this a good design and why?
\\\\
\textbf{Input:} Cornell University is located in Ithaca\\\\
\textbf{Output:} [Cornell University | ORG] is  located in [Ithaca | LOC]}

\answer{}

%%%%%%%%%%%%%%%%%%%%%%%%%%%%%%%%%%%%%%%%%%%%%%

\question{3}{Please give a hypothesis as to why this approach works well in low-resource settings.} 

\answer{}

%%%%%%%%%%%%%%%%%%%%%%%%%%%%%%%%%%%%%%%%%%%%%%

\question{4}{Except for input and output formats, name two differences between the model in this paper and what we used in homework 3.}

\answer{}
%%%%%%%%%%%%%%%%%%%%%%%%%%%%%%%%%%%%%%%%%%%%%%